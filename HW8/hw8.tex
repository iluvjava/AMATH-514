\documentclass[]{article}
\usepackage{amsmath}
\usepackage{amsfonts} 
\usepackage[english]{babel}
\usepackage{amsthm}
\usepackage{mathtools}
\usepackage{bbm}
\usepackage{hyperref}
% \usepackage{minted}
% Basic Type Settings ----------------------------------------------------------
\usepackage[margin=1in,footskip=0.25in]{geometry}
\linespread{1}  % double spaced or single spaced
\usepackage[fontsize=12pt]{fontsize}

\theoremstyle{definition}
\newtheorem{theorem}{Theorem}       % Theorem counter global 
\newtheorem{prop}{Proposition}[section]  % proposition counter is section
\newtheorem{lemma}{Lemma}[subsection]  % lemma counter is subsection
\newtheorem{definition}{Definition}
\newtheorem{remark}{Remark}[subsection]


\hypersetup{
    colorlinks=true,
    linkcolor=blue,
    filecolor=magenta,
    urlcolor=cyan,
}
\usepackage[final]{graphicx}
\usepackage{listings}
\usepackage{courier}
\lstset{basicstyle=\footnotesize\ttfamily,breaklines=true}
\newcommand{\indep}{\perp \!\!\! \perp}
\usepackage{wrapfig}
\graphicspath{{.}}
\usepackage{fancyvrb}

%%
%% Julia definition (c) 2014 Jubobs
%%
\usepackage[T1]{fontenc}
\usepackage{beramono}
\usepackage[usenames,dvipsnames]{xcolor}
\lstdefinelanguage{Julia}%
  {morekeywords={abstract,break,case,catch,const,continue,do,else,elseif,%
      end,export,false,for,function,immutable,import,importall,if,in,%
      macro,module,otherwise,quote,return,switch,true,try,type,typealias,%
      using,while},%
   sensitive=true,%
   alsoother={$},%
   morecomment=[l]\#,%
   morecomment=[n]{\#=}{=\#},%
   morestring=[s]{"}{"},%
   morestring=[m]{'}{'},%
}[keywords,comments,strings]%
\lstset{%
    language         = Julia,
    basicstyle       = \ttfamily,
    keywordstyle     = \bfseries\color{blue},
    stringstyle      = \color{magenta},
    commentstyle     = \color{ForestGreen},
    showstringspaces = false,
}

\begin{document}
\numberwithin{equation}{subsection}
\section*{Notations}
    \begin{itemize}
        \item [1.] $\mathbbm 1_{C}$ to be an indicator set, where $C \subseteq E$, and it's indexed by element $e\in E$ such that $(\mathbbm 1_C)_e = 1$ when $e\in C$ and $0$ when $e\not\in C$
        \item [2.] Define $\delta^+(v):= \{(v, u)\in A| u\in V \}$ to be the set of arcs coming out of the vertex $v$ on the direction graph $D:=(V, A)$. Follows a similar manner, $\delta^-(v):= \{(u, v)| u\in V\}$ be the set of arcs that are coming into the vertex $v$ on the digraph. Similarly, one can define it for a set of vertices as well, which will be a indicator vector representing the set of arcs cutting into or out of a set of vertices on the digraph. 
        \item [3.] Define $\mathbbm 1_{\delta^{\pm}(v)} = \mathbbm 1_{\delta^+(v)}  - \mathbbm 1_{\delta^-(v)}$, which is a vector of $\pm$ denoting arcs that are coming into or out of the vertex $v\in V$. 
    \end{itemize}
\section{Problem 1}
    \begin{prop}
        Let $D:= (V, A)$ be a digraph with $|V| = n$ and $|A| = m$, and define $M_D\in \mathbb R^{n\times m}$ be an incidence matrix of $D$. Then the determinant of any $(n - 1)\times (n - 1)$ sub matrix $M'$ of $M_D$ hs a determinant of $\pm1$ when the chosen columns of $M'$ from $M_D$ forms a tree on the digraph, disregard the directions of the chosen edges. 
    \end{prop}
    \subsection{Proof Strategies}
        For the proof of sufficiency ($\impliedby$), we assume that the submatrix $M'$ has columns of $M_D$ where it corresponds to a cycle: $C$ on the original graph, regardless of directions of the edges. Then, I will show that the absolute values of $\text{det}(M')$ is preseved when I make the directions of edges of $C$ so they aligns; which means that now I can send through a circulation on the cycle, which give me a vector on the null space of $M'$. 
        \par
        For the proof of neccessity ($\implies$), we assume that the sub graph represented by $M'$ is a tree, which implies that each arc must introduce us to a new vertex in the graph, which in the end actually gives us a matrix that is bi-diagonal with nonzeros on the diagonal. 
    \subsection{Proof Direction $\impliedby$}
        WOLG Let $M'$ be an $(n- 1)\times (n - 1)$ sub matrix of $M_D$ that takes $\mathcal C\subset [m]$ columns and $[n-1]$ rows of of $M_D$ ($v_n$ is not chosen to be a row of $M'$) such that they doesn't form a tree on $D$, disregarding the directions of the arcs. Not a tree means columns of $M'$ can contain a cycle if we treat the arcs as edges, for example: 
        \begin{align}
            \underbrace{\text{WOLG let}}_{\text{Read Remark!} }  C := v_0 \underset{a_{k_1}}{\longrightarrow} v_1 \underset{a_{k_2}}{\longleftarrow} v_2 \underset{a_{k_3}}{\longrightarrow} v_3 \cdots v_{l - 1} \underset{a_{k_l}}{\longrightarrow} v_l, \quad l \le n - 1 
        \end{align}
        We want to send a flow to it, because the cycle is subset of arcs represented by $M'$, and if we can send a flow: $\mathbbm 1_C$, then $M'\mathbbm 1_C = \mathbf 0$. The good news is, swapping the direction of any arcs $a_{k_i}$ on $C$ a subgraph of $D$ corresponds to multiplying the $k_i$ column of $M'$ by $-1$, which perserves the absolute value of the determinant. 
        \par
        Consider doing this for all the arcs in $C$ to aling all of them to form a directed cycle for a circulations and we obtained $M''$ as the new matrix, then: 
        \begin{align}
            & |\text{det}(M'')| = |\text{det}(M')|
            \\
            & M'' \mathbbm 1_C = \mathbf 0 \implies |\text{det}(M'')| = 0
            \\
            \implies & |\text{det}(M)| = 0
        \end{align}
        \begin{remark}[A tiny Subtlety here]
           We made the assumption that all the vertices in the cycle $C$ indeed corresponds to the first $(n - 1)$ vertices. This is a legit assumption because if any of the vertices $v_i$ is not in the cycle, them that $t$th row is going to be all zeros! Which trivially makes the matrix having a null space, hence a determinant of zero. 
        \end{remark}

    \subsection{Proof Direction $\implies$}
        WLOG suppose that $(M_D)_{:, 1:(n - 1)}$ is an incidence matrix of a spanning tree so that the first $(n - 1)$ arcs spans a spanning tree in $D$. Further assuming that $M' = (M_D)_{2:n, 1:(n - 1)}$ which is $M_D$ but without the first row. ($v_1$ is not a vertex in our tree... and it's arbitrary.)
        \par
        That vertex in the first row must be connected to a series of arcs: $\{a_{k_1}, a_{k_2}, \cdots, a_{k_l}\}$. Each of those must connect to a different vertex: $v_1\not\in\{v_{j_1}, v_{j_2}, \cdots, v_{j_l}\}$, so it looks like this: 
        \begin{align}
            (M_D)_{\{v_1\}\cup\{v_{j_1}, v_{j_2}, \cdots, v_{j_l}\}, \{k_1, k_2, \cdots, k_l\}}
            \begin{bmatrix}
                \{\pm 1\}_1 &\{\pm 1\}_2  & \cdots& \{\pm 1\}_l
                \\
                \{\mp 1\}_1 &             &       & 
                \\
                &\{\mp 1\}_2     &       & 
                \\
                & &\ddots &
                \\
                & & & \{\mp 1\}_l
            \end{bmatrix}
            \\
            \text{let: }T_l := (M_D)_{\{v_{j_1}, v_{j_2}, \cdots, v_{j_l}\}, \{k_1, k_2, \cdots, k_l\}} 
            = \begin{bmatrix}
                \{\mp 1\}_1 &             &       & 
                \\
                &\{\mp 1\}_2     &       & 
                \\
                & &\ddots &
                \\
                & & & \{\mp 1\}_l
            \end{bmatrix}
        \end{align}
        Observe that, $T_l$ IS A SUBMATRIX OF $M'$ if $T_l = M'$ then we are DONE because it's a diagonal matrix with nonzeros on its diagonal, because $M'$ is a tree and $T_l$ is a sub tree of $M'$, both with $v_1$ missing, I can include another arc and a new vertex incidence to any existing vertex and this new vertex to get $T_{l + 1}$: 
        \begin{align}
            T_{l + 1} &= 
            \begin{bmatrix}
                T_l &  
                \begin{matrix}
                    0 \\ \vdots \\ \pm 1 \\ \vdots \\ 0
                \end{matrix}
                \\
                \begin{matrix}
                    0 & \cdots & 0
                \end{matrix} & \mp 1
            \end{bmatrix}
            =
            \begin{bmatrix}
                T_k & \pm \mathbf e_{i}^{(l)}
                \\
                \mathbf 0 & \mp 1
            \end{bmatrix}
        \end{align}
        Where, we introduce a new $\mp 1$ at the bottom right corner, and a new column to $T_1$ that has exactly one nonzero element in it. This is true because if we haven includes all the arcs yet, then there exists some new arcs that is incidence to existing vertices and it connects to a new vertex that is not in the tree. 
        \par
        Notice that this argument can be applied inductively, we assume that $T_k$ is upper triangular (Base case $T_l$ already is), then the induction holds for all $k< n - 1$, giving us: 
        \begin{align}
            T_{k + 1} &= 
            \begin{bmatrix}
                T_k &  
                \begin{matrix}
                    0 \\ \vdots \\ \pm 1 \\ \vdots \\ 0
                \end{matrix}
                \\
                \begin{matrix}
                    0 & \cdots & 0
                \end{matrix} & \mp 1
            \end{bmatrix}
        \end{align}
        The induction completes with $k = n - 2$, which gives us $T_k = M'$, and $T_k$ is upper triangular with only $\pm 1$ on the diagonal, therefore $\text{det}(T_k) = \pm 1$.
\section{Problem 2}
    \subsection{Problem Statement}
        LP for (50) in the textbook won't work if the objective vector C contains some negative numbers to it. 
    \subsection{Show Strategies}
        I claim to reduce the system of LP  for the Dual of Maxflow to another form that is easier too analyze and show that if any $c_{i, j} < 0, (i, j)\in A$, then the dual problem will become unbounded. 
        \par
        Let $D = (V, A)$ be a digraph with a set of vertices $V$ and a set of arcs $A$. Let's define $M$ be the incidence matrix of the directed graph $G$. Denotes $M'$ to be the incidence matrix of the digraph. Let $c\in \mathbb R^{|A|}$ be a capacity vector. 
    \subsection{Proof}
        The primal formulation of the max cpacity flow is: 
        \begin{align}
            \max\left\lbrace
            \left.
                \langle \mathbbm 1_{\delta^{\pm}(s)}, x\rangle
            \right|
            \mathbf 0 \le x \le c, M'x = \mathbf 0, x\in \mathbb R^{|A|}
            \right\rbrace
        \end{align}
        And after applying duality, we obtain the following dual problem: 
        \begin{align}
            \min\left\lbrace
                \left.\langle c, y\rangle\right|
                y\ge \mathbf 0, y^T+ z^TM'\ge \mathbbm 1_{\delta^\pm(s)}, z\in \mathbb R^{|V| - 2}, y\in \mathbb R^{|A|}_+
            \right\rbrace
        \end{align}
        Let me expand the system out and get: 
        \begin{align}
            & \min \sum_{(i, j)\in A}^{}c_{i,j} y_{i, j}
            \\
            & y_{i, j} + z_{i} - z_{j} \ge 0 \quad \forall\; (i, j)\in A: i\neq 0\wedge j\neq 0
            \\
            & y_{s, j} - z_j \ge \pm 1 \quad \forall\; (i, j) \in \delta^+(s)\cup \delta^-(s)
            \\
            & y_{i, t} + z_i\ge 0 \quad \forall\; j = t \wedge i \neq s
        \end{align}
        Here, the variable $y\ge \mathbf 0$, $z$ is free and I can apply the trick of introducing a new decision variable and a max function. 
        \begin{align}
            & \forall (i, j)\in A: \delta_{i, j} \ge 0
            \\
            y_{i, j} &= \delta_{i, j} + \max(z_j - z_i, 0) \quad \forall\; (i, j)\in A: i\neq 0\wedge j\neq 0
            \\
            y_{s, j} &= \delta_{s, j} + \max(z_j \pm 1, 0) \quad \forall\; (i, j) \in \delta^+(s)\cup \delta^-(s)
            \\
            y_{i, t} &= \delta_{i, t} + \max(-z_i, 0) \quad \forall\; j = t \wedge i \neq s
        \end{align}
        Now, we may consider splitting the objective expression for the miniizations: 
        \begin{align}
            \sum_{(i, j)\in A}^{}c_{i, j}y_{i, j} &= 
            \sum_{(i, j)\in A, i\neq s \wedge j \neq t}^{}  
                c_{i,j}y_{i, j} + 
            \sum_{(s, j)\in A}^{} c_{s, j} y_{s, j} + 
            \sum_{(i \neq s, t)\in A}^{}c_{i, t} y_{i, t}
            \\
            &= 
            \sum_{(i, j)\in A, i\neq s \wedge j \neq t}^{}  
                c_{i, j}(\delta_{i, j} + \max(z_j - z_i, 0))\cdots 
            \\
            & + \sum_{(s, j)\in A}^{} c_{s, j} (\delta_{s, j} + \max(z_j \pm 1, 0))\cdots
            \\
            & + \sum_{(i \neq s, t)\in A}^{}c_{i, t}(\delta_{i, t} + \max(-z_i, 0))
        \end{align}
        Notice that, we can factor out the term $\sum_{(i, j)\in A}c_{i, j}\delta_{i, j}$, in which case if any of the $c_{i, j}\le 0$, we can make it unbounded for any feasible solution of $y, z$ by increasing values of $\delta$ indefinitely. 
    \subsection{An Example}
        There is no example for Maxflow, only the Mincut, because the primal is infeasible due to the constraints $x\ge \mathbf 0$. The polytope is empty. 
        \par
        Consider a graph that has only $s \underset{c <  0}{\longrightarrow} t$ to it, then the dual is: $\min\{cy: y \ge \mathbf 0 , y \ge 1\}$, decision variable $z$ is gone because the graph only has $\{s, t\}$ as the vertex set. Obivously it's unbounded when $c < 0$. 
        \begin{remark}
            If people want to add negative capacity to model flow in reverse direction, please consider adding parallel arcs in opposite direction with positive capacity between those 2 vertices instead.
        \end{remark}

\section{Problem 3}
    \subsection{Problem Statement}
        Explain why the model in Application 4.4 on pp 73 works. 
    \subsection{The Setting up of The Graph}
        Putting the citites having surplus, deficit into a bipartite directed graph $D = (U\dot\cup W, A)$. Let $U$ be the cities of curplus and $W$ be the cities of deficit in freighters. Connects every $u\in U$ to every $w\in V$ by an arc with infinite capacity, going from $U$ to $W$, associate cost with each arc: $(u_i, w_j)$ by the shortest distance between these 2 cities of surplus and deficit. Let $k: A \mapsto \mathbb R_+$ be our cost function, and $c: A\mapsto \mathbb R_+$ be our capacity function. Mathematically: 
        \begin{align}
            & U:: \text{FreightersSurplus Cities}
            \\
            & W::\text{Freighers Deficit Cities}
            \\
            & \forall u\in U, w\in W
            \\
            & c((u, w)) := +\infty
            \\
            & k((u, w)) := \text{min\_distance}(\text{city: u}, \text{city v})
        \end{align}
        \par
        Next, we introduce auxilary vertex $s, t$ as the source and the sink vertex for the flow. Construct arcs from $s$ to all vertices in $U$, cities with surplus, with capacity equals to the empty freighters in that each city. $(s, u)\in E, u\in U$, $c((s, u)) = \sigma_u$, connecting each vertex $w\in W$, cities with deficit in freighters to $t$, with an edge $(w, t)$ with a capacity equals to the deficit of freighters for that city: $\nu$. 
        \begin{align}
            & c((s, u)) := \sigma_{s, u}, k((s, u)) := 0 \quad \forall u \in U
            \\
            & c((w, t)) := \nu_{w, t}, k((w, t)) := 0\quad \forall w\in W
        \end{align}
        \par
        Take note that from the original problem, the total number of deficit among all cities and the surplus adds up to zero. A maximum flow will have a flow value equals to the total number of surplus/deficit among all cities: $\text{value}(f) = \sum_{i= 1}^{|U|}\sigma_i = \sum_{i = 1}^{|W|}\nu_i$. The Mincut won't include any arcs going between $U, W$ because they have infinite capacity. This implies that for all $u\in U$, the flow cutting out of it equals the total surplus of that city, it also means that the flow coming out of each cities $w\in W$ equals to the defecit $\nu_w$. Flow $f$ is also going to be extreme because it's the minimum cost flow. 
        \par
        Because flows in equals flow out therefore all the freighters in $U$ are routed to $W$, and all edges connecting $s, U$ and $W, t$ are saturated, meaning that the flow takes away all surplus of freighters in $U$ to $W$ and filled up all the defecit. 
        \par
        A maximum flow from the cities of surplus to deficit is a fixed flow problem, and to minimize the costs will minimizes the distance travelled perunit freighters. Let's just stick with real numbers for now which (Cause in the real world, we can't cut a frieghters and send it to different cities).
    \subsection{Why Bipartite Graph, Why Only Transport Between Cities of Freighters Surplus and Deficit? }
        Triangle inequality, sending some freighters from city $u_i$ to $u_j$ then to $w_k$ is always longer than sending directly from $u_i$ to $w_k$.\footnote{Triangule Inequality holds for noneuclidean geometry: The earth surface.}
    \subsection{Why is Maxflow Best Routing?}
        Because given any transportation function $g: A\mapsto \mathbb R_+$ for the full graph of cities: $G = (V, A)$, we can construct the loss function for each vertex: $\text{loss}(v):=-\sum_{a\in \delta^+(v)}^{} g(a) + \sum_{a \in \delta^-(v)}^{}g(a)$, and we minimize: $\text{loss}(V) := \sum_{v\in V}^{}|\text{loss}(v)|$. 
        \par
        Then, we can define a new bipartite graph where $U$ corresponds to all $v$ such that $\text{loss}(v) > 0$, and $W$ corresonds to all $v$ such that $\text{loss}(v) < 0$, and we follows the exact same setup as application 4.4. 
        \par
        The solution will then reduce $\text{loss}(V)=0$, which gives us a circulations. And that circulations will have least amount of cost for transporting extra empty freighters between cities. 
        
\section{Problem 4}
    This is the table: 
    $$
    \begin{array}{|l|l|l|l|l|l|l|l|l|l|}
        \hline & \text { Team } & \text { Wins } & \text { Left to Play } & \text { vs. A } & \text { vs. B } & \text { vs. C } & \text { vs. D } & \text { vs. E } \\
        \hline \text { A } & \text { Houston Astros } & 75 & 28 & * & 3 & 8 & 7 & 3 & \\
        \hline \text { B } & \text { Los Angeles Angels } & 71 & 28 & 3 & * & 2 & 7 & 4 & \\
        \hline \text { C } & \text { Oakland Athletics } & 69 & 27 & 8 & 2 & * & 0 & 0 & \\
        \hline \text { D } & \text { Seattle Mariners } & 63 & 27 & 7 & 7 & 0 & * & 0 & \\
        \hline \text { E } & \text { Texas Rangers } & 49 & 27 & 3 & 4 & 0 & 0 & * & \\
        \hline
        \end{array}
    $$
    Unfortunately, Texas Rangers can't win with first place anymore. Assuming that Texas Rangers (Team E) wins all the remaining races, they they had $49 + 27 = 76$ wins. Sin Team E already won all, they thre is no matches between any other teams with team $E$ any more. 
    \par
    Other teams: $A, B, C, D$ are left to play with each other for certainty for the remaining seasons. $A$ has  AT LEAST $18$ unfinished races, B has $12$, C has $10$ and D has 14. How many winning will happen for races between: A, B, C, D in total? It's: 
    \begin{align}
        (75 + 71 + 69 + 63) + (18 + 12 + 10 + 14) = 332
    \end{align}
    Ok, on average, how many are wins these 4 teams getting in total? It's: $332/4 = 83$. Ok, that means the team of most wining among A, B, C, D is wining at least $83$ times. Team E won't get first place for certainty, because team E can't win more than $75$. 
    \par
    More matches and races will only add more to these 3 teams, making the higher even higher, and maximum of winning team E can get is 75, which is lower than 83. 


\end{document}