\documentclass[]{article}
\usepackage{amsmath}
\usepackage{amsfonts} 
\usepackage[english]{babel}
\usepackage{amsthm}
\usepackage{mathtools}
\usepackage{hyperref}
% \usepackage{minted}
% Basic Type Settings ----------------------------------------------------------
\usepackage[margin=1in,footskip=0.25in]{geometry}
\linespread{1}  % double spaced or single spaced
\usepackage[fontsize=12pt]{fontsize}

\theoremstyle{definition}
\newtheorem{theorem}{Theorem}       % Theorem counter global 
\newtheorem{prop}{Proposition}[section]  % proposition counter is section
\newtheorem{lemma}{Lemma}[subsection]  % lemma counter is subsection
\newtheorem{definition}{Definition}
\newtheorem{remark}{Remark}[subsection]


\hypersetup{
    colorlinks=true,
    linkcolor=blue,
    filecolor=magenta,
    urlcolor=cyan,
}
\usepackage[final]{graphicx}
\usepackage{listings}
\usepackage{courier}
\lstset{basicstyle=\footnotesize\ttfamily,breaklines=true}
\newcommand{\indep}{\perp \!\!\! \perp}
\usepackage{wrapfig}
\graphicspath{{.}}
\usepackage{fancyvrb}

%%
%% Julia definition (c) 2014 Jubobs
%%
\usepackage[T1]{fontenc}
\usepackage{beramono}
\usepackage[usenames,dvipsnames]{xcolor}
\lstdefinelanguage{Julia}%
  {morekeywords={abstract,break,case,catch,const,continue,do,else,elseif,%
      end,export,false,for,function,immutable,import,importall,if,in,%
      macro,module,otherwise,quote,return,switch,true,try,type,typealias,%
      using,while},%
   sensitive=true,%
   alsoother={$},%
   morecomment=[l]\#,%
   morecomment=[n]{\#=}{=\#},%
   morestring=[s]{"}{"},%
   morestring=[m]{'}{'},%
}[keywords,comments,strings]%
\lstset{%
    language         = Julia,
    basicstyle       = \ttfamily,
    keywordstyle     = \bfseries\color{blue},
    stringstyle      = \color{magenta},
    commentstyle     = \color{ForestGreen},
    showstringspaces = false,
}



\begin{document}
\numberwithin{equation}{subsection}
\section{Theorems from Classes}

\section{Problem 3.3}
    \begin{prop}
        The number of non-zero elements you can put into a matrix such that all columns and row of the matrix has no more than 2 elements equals to the minimal number of lines (going horizontally or vertically) on the matrix such that it covers all the non-zero elements. 
    \end{prop}
    \begin{proof}
        Suppose that the matrix $A$ is a $m\times n$ matrix. We firstly need to represent the non-zero elements in the matrix $A$ as edges in the bi-partite graph, and the a row or a column of the matrix as a vertex that in that bipartite graph that covers some of the edges in the bipartite graph. 
        \par
        Define bi-partite graph $G = (V \dot{\cup}V', E)$, where $\dot{\cup}$ is the disjoint union of 2 sets, and we establish notations: 
        \begin{align}
            & V := \{v_i\}_{i = 1}^{n}
            \\
            & V' := \{v_j'\}_{j = 1}^{m}
        \end{align}
        Each set of the bipartite graph represents the row index and column index of the matrix. Then, we set the correspondence of a non-zero element in the matrix as an edge going between $V', V$, like this: 
        \begin{align}
            
        \end{align}
    \end{proof}
\section{Problem 3.5}
\section{Problem 3.11}
\section{Problem 3.17}
\end{document}