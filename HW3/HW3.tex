\documentclass[]{article}
\usepackage{amsmath}\usepackage{amsfonts}
\usepackage[english]{babel}
\usepackage{amsthm}
\theoremstyle{definition}
\newtheorem{theorem}{Theorem}
\newtheorem{prop}{Proposition}
\newtheorem{lemma}{Lemma}
\newtheorem{definition}{Definition}
\usepackage[margin=1in,footskip=0.25in]{geometry}
\usepackage{mathtools}
\usepackage{hyperref}
\hypersetup{
    colorlinks=true,
    linkcolor=blue,
    filecolor=magenta,
    urlcolor=cyan,
}
\usepackage[final]{graphicx}
\usepackage{listings}
\usepackage{courier}
\lstset{basicstyle=\footnotesize\ttfamily,breaklines=true}
\newcommand{\indep}{\perp \!\!\! \perp}
% \usepackage{wrapfig}
\graphicspath{{.}}
% \usepackage{fancyvrb}

%%
%% Julia definition (c) 2014 Jubobs
%%
\usepackage[T1]{fontenc}
\usepackage{beramono}
\usepackage[usenames,dvipsnames]{xcolor}
\lstdefinelanguage{Julia}%
  {morekeywords={abstract,break,case,catch,const,continue,do,else,elseif,%
      end,export,false,for,function,immutable,import,importall,if,in,%
      macro,module,otherwise,quote,return,switch,true,try,type,typealias,%
      using,while},%
   sensitive=true,%
   alsoother={$},%
   morecomment=[l]\#,%
   morecomment=[n]{\#=}{=\#},%
   morestring=[s]{"}{"},%
   morestring=[m]{'}{'},%
}[keywords,comments,strings]%

\lstset{%
    language         = Julia,
    basicstyle       = \ttfamily,
    keywordstyle     = \bfseries\color{blue},
    stringstyle      = \color{magenta},
    commentstyle     = \color{ForestGreen},
    showstringspaces = false,
}

\linespread{1}
\usepackage[fontsize=12pt]{fontsize}
\begin{document}
\numberwithin{equation}{subsection}
\begin{center}
    AMATH 514 SPRING 2022 HONGDA LI HW3
\end{center}
\section{Problem 2.16}
    \begin{prop}
        \begin{align}
            (\exists x \ge \mathbf 0: Ax = \mathbf 0 )
            \iff 
            (y^TA \ge \mathbf 0 \implies y^TA = \mathbf 0)
        \end{align}
    \end{prop}
    \noindent
    Introduce the lemma: 
    \begin{lemma}
        if $x >\mathbf 0$, and $y\ge \mathbf 0$, then $\langle x, y\rangle \ge \mathbf 0$. 
    \end{lemma}
    \begin{proof}[Lemma 1]
        This is true because $\langle x, y\rangle = \sum_{i = 1}^{n}x_i y_i \ge \mathbf 0$, we are just multiplying each of the inequality by a non-negative number and then sum then all up. 
    \end{proof}
    \begin{proof}[Proposition 1]
        Proof of sufficiency $\implies$: 
        \begin{align}
            &\text{choose }x \text{ s.t: } Ax = \mathbf 0 ,\; x > \mathbf 0
            \\
            & y^TA \ge\mathbf 0 \wedge y^TAx = \mathbf 0 \implies y^TA = \mathbf 0
        \end{align}
        $y^TA \ge \mathbf 0$ and by $y^TAx = 0, x\ge 0$, we know that $y^TA = \mathbf 0$, because you can't sum up positive number and still get zero. 
        \par
        Proof of neccessity $\impliedby$: we will use prove by contradiction, we assume that $y^TA\ge \mathbf 0$ and $y^TA = \mathbf 0$, and for contradiction we assume $\not\exists x > \mathbf 0: Ax = \mathbf 0$. 
        \begin{align}
            & y^TA = \mathbf 0 \implies y^TAx = \mathbf 0 \quad \forall x
            \\
            &\hspace{1.1em}
            \exists x > \mathbf 0: \underbrace{y^TA}_{\ge \mathbf 0} \underbrace{x}_{x > 0} = \mathbf 0
            \\
            & \text{Contradicts Lemma 1}
        \end{align}

    \end{proof}

\section{Problem 2.21}
    \begin{prop}
        If the polytope $P:= \{A| Ax \le b\}\neq \emptyset$, prove $x^+: x^+ = \max \{c^Tx | Ax \le b\}$ is attained by an vertex $x^+\in P$. 
    \end{prop}
    \par
    Here is the approach for this problem. A polytope is closed therefore the objective value is going to be bounded. Next, if supremum of the objective exists then there is a point inside of the closed polytope $p$ that attains it. 
    \par
    To show that the pint $x^+$ is a vertex, we assume it's not, then we show that either we can wiggle it around to improve $\langle c, x\rangle$, or we can just wiggle it so it becomes an vertex in $P$ eventually, hence it has to be a vertex. 

\section*{2.26}

\section*{2.27}


\end{document}