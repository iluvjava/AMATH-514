\documentclass[]{article}
\usepackage{amsmath}
\usepackage{amsfonts} 
\usepackage[english]{babel}
\usepackage{amsthm}
\usepackage{mathtools}
\usepackage{hyperref}
% \usepackage{minted}
% Basic Type Settings ----------------------------------------------------------
\usepackage[margin=1in,footskip=0.25in]{geometry}
\linespread{1}  % double spaced or single spaced
\usepackage[fontsize=12pt]{fontsize}

\theoremstyle{definition}
\newtheorem{theorem}{Theorem}       % Theorem counter global 
\newtheorem{prop}{Proposition}[section]  % proposition counter is section
\newtheorem{lemma}{Lemma}[subsection]  % lemma counter is subsection
\newtheorem{definition}{Definition}
\newtheorem{remark}{Remark}[subsection]


\hypersetup{
    colorlinks=true,
    linkcolor=blue,
    filecolor=magenta,
    urlcolor=cyan,
}
\usepackage[final]{graphicx}
\usepackage{listings}
\usepackage{courier}
\lstset{basicstyle=\footnotesize\ttfamily,breaklines=true}
\newcommand{\indep}{\perp \!\!\! \perp}
\usepackage{wrapfig}
\graphicspath{{.}}
\usepackage{fancyvrb}

%%
%% Julia definition (c) 2014 Jubobs
%%
\usepackage[T1]{fontenc}
\usepackage{beramono}
\usepackage[usenames,dvipsnames]{xcolor}
\lstdefinelanguage{Julia}%
  {morekeywords={abstract,break,case,catch,const,continue,do,else,elseif,%
      end,export,false,for,function,immutable,import,importall,if,in,%
      macro,module,otherwise,quote,return,switch,true,try,type,typealias,%
      using,while},%
   sensitive=true,%
   alsoother={$},%
   morecomment=[l]\#,%
   morecomment=[n]{\#=}{=\#},%
   morestring=[s]{"}{"},%
   morestring=[m]{'}{'},%
}[keywords,comments,strings]%
\lstset{%
    language         = Julia,
    basicstyle       = \ttfamily,
    keywordstyle     = \bfseries\color{blue},
    stringstyle      = \color{magenta},
    commentstyle     = \color{ForestGreen},
    showstringspaces = false,
}



\begin{document}
\numberwithin{equation}{subsection}
\begin{center}
    AMATH 514 SPRING 2022 HONGDA LI HW3
\end{center}
\section{Problem 2.16}
    \begin{prop}
        \begin{align}
            (\exists x > \mathbf 0: Ax = \mathbf 0 )
            \iff 
            (y^TA \ge \mathbf 0 \implies y^TA = \mathbf 0)
        \end{align}
    \end{prop}
    \begin{proof}[Proposition 1]
        Proof in direction $\implies$: 
        \begin{align}
            &\text{choose }x \text{ s.t: } Ax = \mathbf 0 ,\; x > \mathbf 0
            \\
            & y^TA \ge\mathbf 0 \wedge y^TAx = \mathbf 0 \implies y^TA = \mathbf 0
        \end{align}
        $y^TA \ge \mathbf 0$ and by $y^TAx = 0, x> \mathbf 0$, we know that $y^TA = \mathbf 0$, because you can't sum up positive number and still get zero. Geometrically it's saying that if $\mathbf 0$ is in the affine hull of the cone made by the columns of $A$, and, there is a vector $y$ such that every columns of $A$ falls to one side of $y$, then it must be the case that the columns of $A$ are on the hyperplane perpendicular to $y$. 
        \\
        \noindent
        For the proof in the direction $\impliedby$ we consider a direct proof starting with: 
        \begin{align}
            & (y^TA \ge \mathbf 0 \implies y^TA= \mathbf 0)
            \implies 
            (\exists x \ge \mathbf 0: Ax = \mathbf 0)
            \\
            & \neg(\exists y^TA \ge 0 \wedge y^TA \neq \mathbf 0)
            \implies 
            (\exists x \ge \mathbf 0: Ax = \mathbf 0)
        \end{align}
        Take note that if $y^TA \ge \mathbf 0 \wedge y^TA \neq \mathbf 0$, assuming it's true then it can be simplified into $y^TA \ge \mathbf 0 \wedge y^TA > \mathbf 0$. This is saying that if there doesn't exist any hyperplane defined by $y$ such that it separates the columns of $y$ onto the positive side and no columns are on the hyperplane, then they have to be all lay on the strict positive side of $y$. Then we can consider the statement for each of the column $a_i$ of $A$ making it looks like one of the cases of Ferkas's Lemma: 
        \begin{align}
            & \forall i \in [n]: \neg(\exists y^TA \ge \mathbf 0 \wedge y^Ta_i > 0)
            \\ 
            &\quad \implies 
            \forall i \in [n]: \exists z_i \ge \mathbf 0: Az_i = -a_i
        \end{align}
       In this case, we are setting the vector $b = -a_i$ and then applying the ferkas's lemma, attempting to crate a cone usin $z_i$ scaling columns of $A$ to include $-a_i$. Which then implies the fact that: 
       \begin{align}
            & A\left(
                \sum_{j = 1}^{n}z_i
            \right) = -A\mathbf 1
            \\
            & 
            A\left(
                \underbrace{\sum_{j = 1}^{n}z_i + e_i}_{> \mathbf 0}
            \right) = \mathbf 0
       \end{align}
       Here,we summed up the results from (1.0.7) and then move then to one side of the equation, then the quantity multipled by matrix $A$ is strictly positive because $z_i\ge \mathbf 0$ at and the sum of $e_i$ are all strictly positive. 
    

    \end{proof}

\section{Problem 2.21}
    \begin{prop}
        If the polytope $P:= \{A| Ax \le b\}\neq \emptyset$, prove $x^+: x^+ = \max \{c^Tx | Ax \le b\}$ is attained by an vertex $x^+\in P$. 
    \end{prop}
    \par
    Here is the approach for this problem. A polytope is closed therefore the objective value is going to be bounded. Next, if supremum of the objective exists then there is a point inside of the closed polytope $p$ that attains it. 
    \par
    To show that the pint $x^+$ is a vertex, we assume it's not, then we show that either we can wiggle it around to improve $\langle c, x\rangle$, or we can just wiggle it so it becomes an vertex in $P$ eventually, hence it has to be a vertex. 
    \begin{proof}
        Let the set $\mathcal I$ be the indices for all the tight constraints for the given point $x^+ \in P$, then $\mathcal I := \{i\in [m]: a_i^Tx^+= b_i\}$ where $a_i$ is a vector denoting the ith row of matrix $A$, and $|\mathcal I| \le n$. Next we consider the wiggle vector $w \in \mathbb R^n$ such that $w \neq \mathbf 0$ for $x^+$. By the definition that $x^+$ is not a vertex, we know that the sub matrix $A_{\mathcal I, :}$ whose rows are indexed by $\mathcal I$ has $\text{rank}(A_{\mathcal I, :})\le n$. Therefore: 
        \begin{align}
            & \exists w \neq \mathbf 0 : (A_{\mathcal I, :})w = \mathbf 0
            \\
            &\hspace{1.1em} \implies
            \forall j \in [m]\setminus \mathcal I: 
            a_j^Tx^+ < b_j
        \end{align}
        The inequality is loose, therefore we can try inserting the quantity $\alpha_ja_j^Tw$ into the inequality like: 
        \begin{align}
            & \exists\; \alpha_j > 0 : a_j^Tx^+ \pm \alpha_ja_j^Tw \le b_j
            \\
            &\hspace{1.1em} \implies 
                \pm \alpha_j a^T_jw \le b_j - a_j^Tx^+
            \\
            &\hspace{1.1em} \implies 
                -(b_j - a_j^Tx^+)\le \alpha_j a^T_jw \le b_j - a_j^Tx^+ 
            \\
            &\hspace{1.1em} \implies 
            \alpha_j
            |a_j^Tw|\le |b_j - a_j^Tx^+|
            \\
            &\hspace{1.1em} \implies 
            \alpha_j \le \left|
                \frac{b_j - a_j^Tx^+}{a_j^Tw}
            \right|
        \end{align}
        Now, we can choose the minimal $\alpha_j$ to determine how much wiggle room we have for $x^+$ such that it stills remains in $P$, name that variable $\alpha$ and it would be given as: 
        \begin{align}
            & \alpha := \min_{j \in [m]\setminus \mathcal I}\left\lbrace
                \left|\frac{b_j - a_j}{a^T_jw}\right|
            \right\rbrace
            \\&\quad\implies
            A(x^+ \pm \alpha w)  \le b
            \implies x^+ \pm \alpha w \in P
        \end{align}
        Next, we may consider the objective value achieve by this wiggled point: 
        \begin{align}
            &\langle c, x^+ \pm \alpha w\rangle
            \\
            & \langle c,\alpha w\rangle \neq 0 
            \\&\quad \implies
            (\langle c, x^+ + \alpha w\rangle > \langle c, x^+\rangle)
            \vee 
            (\langle c, x^+  -\alpha w\rangle > \langle c, x^+\rangle)
        \end{align}
        Therefore, in this case, we $x^+$ cannot be an optimal yet, which is a contradiction. Otherwise, it has to be the case that $\langle c, \alpha w\rangle = 0$, which tells use that we can make a new point $x^{++}$ be either $x^+ + \alpha w$ or $x^+ - \alpha w$, and then we will have one more tight constraint for the set $\mathcal I$. More specifically, choose: 
        \begin{align}
            & \forall j^+\in \arg\min_{j \in [m]\setminus \mathcal I}
            \left\lbrace
                \left|\frac{b_j - a_j}{a^T_jw}\right|
            \right\rbrace
            \\& \quad \implies
            (\alpha_{j^+} x^+ + \alpha_{j^+}a^T_{j^+}w = b_i )
            \vee 
            (\alpha_{j^+} x^+ - \alpha_{j^+}a^T_{j^+}w = b_i )
        \end{align}
        Choose the right sign for $\alpha_{j^+}$ such that it makes the constraints $j^+$ tight, then we have one or more more tight constraints for our point $x^+$, which means that: $A_{\mathcal I \cup \{j^+\}, :}$ must have a higher rank than $A_{\mathcal I, :}$. To this regard, we can repeat this process by redefining $x^+ = x^{++}, \mathcal I := \mathcal I \cup \{j^+\}$, and repeat the proof. Eventually, we will have $x^+$ becoming a vartex because $\text{rank}(A_{\mathcal I, :}) = n$ eventually $x^+$ will be a vertex in $P$. This is true because polytope is closed, it's impossible that $x^+$ gives us some constraints that is unbounded. 

    \end{proof}
\section*{2.26}
    Find a case where $\{x|Ax \ge b\}$ and $\{y| y\ge \mathbf 0: y^TA = c^T\}$ are empty.
    \par
    Consider 2d where we have $x_1, x_2$ as coordinates for primal and $y_1, y_2$ for dual: 
    \begin{align}
        &
       \underbrace{ \begin{bmatrix}
            -1 & 0
            \\
            1 & 0
        \end{bmatrix}}_{A}
        \begin{bmatrix}
            x_1 \\ x_2 
        \end{bmatrix} \le 
        \begin{bmatrix}
            -1 \\ -1
        \end{bmatrix}
        \\
        & 
        c = \begin{bmatrix}
            0 \\ 1
        \end{bmatrix}
    \end{align}
    Observe that $c$ is in the null space of $A$, therefore it's impossible that $y^TA = c^T$, at the same time the polytope is empty too. 
\section*{2.27}
    \begin{prop}
        define the primal and dual problem pairs for LP to be the following: 
        \begin{align}
            \max\{\langle c, x\rangle: Ax \le b\} \le 
            \min\{
                y^Tb: y^TA = c^T, y \ge \mathbf 0    
            \}
        \end{align}
        Both $x, y$ are optimal iff $\forall i \in [m]: y_i = 0\implies a_i^Tx = b_i$. The weights on the loose constraints will have to be zero. 
    \end{prop}
    \begin{proof}
        \begin{align}
            & Ax \le b
            \\
            & \quad \implies
            Ax + s = b \quad s \ge \mathbf 0    
            \\
            & y^T(Ax + s) = y^Tb
            \\
            & \underbrace{y^TA}_{=c^T} x + \underbrace{y^Ts}_{\ge \mathbf 0} = y^Tb
            \\
            & c^Tx + y^Ts = y^Tb
        \end{align}
        Take note that $y^Tb = c^Tx$ by strong duality (we proved that part in class), therefore $y^Ts = 0$. However, take note that whenever index $j$ are of a loose constraints for the optimal $x\in P$, $a_j^Tx \le b_j \implies s_j > 0$, by the fact that $y\ge \mathbf 0, s \ge \mathbf 0$, it must be the case that $y_j = 0$ for all such $j$. Proof done. 
    \end{proof}


\end{document}